\section{Experimentacion Adicional}
\subsection{Experimentaciones adicionales a Tailandia}
En la carpeta \textbf{/www.asianust.ac.th experiment/} se encuentra un experimento adicional realizado con una universidad de Tailandia como destino, no lo presentamos con detalle en el informe porque nos parece redundante ya que los resultados fueron similares a los experimentos presentados en este informe. En este ejemplo la deteccion de enlaces intercontinentales con la herramienta geografica arrojo resultados esperados, pero el ocultamiento de hops hizo que la herramienta estadistica fallara al detectar \textbf{(ver Traceroute a www.asianust.ac.th)} el salto de Argentina a Estados Unidos, pero fue detectado sin problemas el salto entre Estados Unidos y Tailandia.

\subsection{Experimentaciones adicionales a Australia}
En la carpeta \textbf{www.uq.edu.au.txt experiment} se encuentra un experimento adicional realizado con una universidad de Australia como destino, no lo presentamos con detalle en el informe porque nos parece redundante ya que los resultados fueron similares a los experimentos presentados en este informe. En este ejemplo observamos nuevamente cosas extrañas en los saltos:
\begin{enumerate}
	\item Uruguay, \textbf{us-cl.redclara.net.}
	\item Uruguay, \textbf{i2-us.redclara.net.}
	\item United States, Ann Arbor, \textbf{et-1-0-0.105.rtr.hous.net.internet2.edu.}
\end{enumerate}
Donde el incremento se encuentra en los dos primeros hosts y no entre los dos ultimos, observando el nombre del host marcado en \textbf{negrita} vemos nuevamente como en un experimento anterior que el cable de larga distancia se oculta con IPs Uruguayas en ambos extremos.(Notar el \textbf{us} del hostname del 2do host.)

\subsection{Nota sobre herramienta Geografica}
Cuando decimos que la herramienta arrojo datos esperados en los casos donde ambos extremos del cable submarino tiene un solo pais y se desfasa en un hop el salto dibujado en el mapa nos referimos a que se detecta correctamente el salto entre continentes, no asi el hop exacto donde ocurren.

\subsection{Resultados extraños}
De ciertas experimentaciones adicionales realizadas por curiosidad y en el transcurso del desarrollo del software necesario para este trabajo, pero no presentadas en este informe dado el alcance del trabajo obtuvimos resultados extraños, como servidores de ISP intermedios contestando \texttt{Echo Reply} como si fueran el host destino, creemos que esto se debe a un cacheo intermedio que realiza el ISP para minimizar la redireccion de trafico fuera del pais. Otros experimentos realizados sobre la red movil 3g anclando la conexion a una pc, mostraron una cantidad de saltos muy grande dentro del pais, con tiempos de \texttt{Round Trip Time} muy altos en comparacion al enlace submarino hasta el host destino, los nodos distinguidos de la ruta quedaron enmarcados dentro del pais, lo cual podria explicarse por el nivel de congestion de la red de telefonia de la capital. Otros experimentos realizados sobre redes 3G arrojan una cantidad innecesaria de hops entre continentes para llegar a destino, particularmente los paquetes iban de Argentina, a Uruguay, luego a Italia, luego a Estados Unidos, volvian al Reino Unido, y finalmente a destino en europa.\\
Algunos de los datos de los experimentos mencionados pueden encontrarse en el comprimido adjunto por fuera del informe.