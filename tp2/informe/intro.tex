\section{Introducción}
Los objetivos de este trabajo practico son varios, para la primera etapa de desarrollo del trabajo, realizamos una herramienta que realiza un traceroute utilizando el protocolo \texttt{ICMP} con \texttt{TTL} incrementales, utilizando scapy en python. Luego de tener la primera version de la herramienta, se le agregaron funcionalidades, entre ellas el calculo del RTT entre hops, un analisis estadistico de estos datos, incluyendo el calculo de promedio, desv. estandar y \texttt{zscore}. Adem\'as, dado que la segunda etapa del trabajo implica el analisis de las trazas a diferentes continentes y el descubrimiento de nodos distinguidos, particularmente enlaces submarinos, se utilizaron varias APIs p\'ublicas para la obtencion del nombre del host del hop a partir de su ip(\texttt{DNS Reverse Lookup}), asi como tambien la \texttt{geolocalizacion} de la IP, tanto en terminos de pais y ciudad, como tambien en terminos de latitud y longuitud en un planisferio. Para complementar los analisis, se realizaron graficos, de distribucion de las mediciones estadisticas y un grafico sobre un planisferio que muestra la traza por los diferentes hops distribuidos en el mundo.