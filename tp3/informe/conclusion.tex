\section{Conclusiones}
A lo largo del trabajo observamos que es factible decidir experimentalmente algunos parámetros en el protocolo de manera de llegar al funcionamiento deseado. En una primera instancia estimamos valores de $\alpha$ y $\beta$ que hacen a una estimación mejor del RTO. Una vez encontrados los valores buscados, medimos, usando esos valores, el comportamiento del protocolo en relación a distintos estados y características de la red. De los análisis develamos algunas relaciones como que la cantidad de paquetes transmitidos de forma errónea  (luego retransmitidos) aumenta al aumentar el delay introducido al enviar los paquetes ACK y la probabilidad de pérdida de paquetes. Esto quiere decir que, como es esperado, una red que está congestionada impacta en forma negativa en el desempeño del protocolo. Si bien la relación anterior es constante entre los valores de $\alpha$ y $\beta$ elegidos, para alguno de estos valores, pequeños cambios en delay y probabilidad de pérdida de paquetes no provocan grandes cambios en la cantidad de retransmiciones; por ejemplo en $\alpha:0.9$ y $\beta:0.5$. Esto implica que si el protocolo está fucnionando en una red con las propiedades anteriores que fluctúan mucho, esta elección de parámetros son más apropiados. En cambio, con  $\alpha:0.9$ y $\beta:0.9$, los datos obtenidos son muy variables.\\
\par En cuanto al RTO, los experimentos realizados no nos dan mucha información para elegir $\alpha$ y $\beta$, ya que los datos obtenidos son similares. Podemos ver que la precisión medida es mas o menos la misma para distintos $\alpha$ y $\beta$. Aún así, los datos muestran que la precisión es mucho más constante (ante variaciones en el delay y la probabilidad de error) para los valores  $\alpha:0.9$ y $\beta:0.5$ y  $\alpha:0.9$ y $\beta:0.9$.
\par Considerando lo anterior, eligiríamos los valores  $\alpha:0.9$ y $\beta:0.5$ para el protocolo.

