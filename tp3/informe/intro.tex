\section{Introducción}
En este trabajo pr\'actico ejercitaremos las nociones del nivel de transporte estudiadas en la materia a
través del análisis de un protocolo PTC implementado por la c\'atedra. Este protocolo se encuentra basado en TCP, pero simplficado. Es bidireccional, orientado a conexi\'on y confiable. 

La idea b\'asica es generar un escenario de an\'alisis sobre el cual estudiaremos los par\'ametros $\alpha$ y $\mathcal{B}$ que optimizan el c\'aluclo de RTO (Retransmission Time Out). Para ello definiremos un esquema, en el contexto de una red local, de cliente-servidor en el cual el cliente env\'ia datos y el servidor solo los reconoce. Estudiaremos de qu\'e manera afectan los efectos de red (delay y p\'erdida de paquetes) en los par\'ametros encontrados realizando diversas experimentaciones.