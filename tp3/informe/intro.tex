\section{Introducción}
En este trabajo pr\'actico ejercitaremos las nociones del nivel de transporte estudiadas en la materia a
través del análisis de un protocolo PTC implementado por la c\'atedra. Este protocolo se encuentra basado en TCP, pero simplificado. Es bidireccional, orientado a conexi\'on y confiable.\\

El objetivo es generar un escenario de an\'alisis en el contexto de una red local, de cliente-servidor en el cual el cliente env\'ia datos y el servidor solo los reconoce. En una etapa inicial, realizaremos diversos experimentos y observacion sobre esos resultados para fijar valores optimos para los par\'ametros $\alpha$ y $\mathcal{B}$ que optimizan el c\'alculo de RTO (Retransmission Time Out).\\

Finalmente, analizaremos de qu\'e manera afectan los efectos de red (delay y p\'erdida de paquetes) a la aproximacion del RTO con los parametros de la etapa inicial fijos.