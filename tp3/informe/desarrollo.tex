\section{Desarrollo}
Para poder realizar una experimentaci\'on adecuada sobre el protocolo PTC es necesario testear el mismo en el contexto de una red local, en donde el delay y la p\'erdida de paquetes afectan a la transmisi\'on de los datos. Para ello realizamos algunos cambios en la implementaci\'on del protocolo.

Para introducir latencia en el env\'io de paquetes modificamos el archivo 'protocol.py' de ptc. La idea es que cada vez que se est\'a por enviar un paquete en el m\'etodo $'send\_and\_queue'$ se llame al m\'etodo $'time.sleep(ACK\_delay)'$ que suspende la ejecuci\'on por una cantidad arbitraria de segundos determinado por el cliente justo antes de enviar datos al servidor.

De manera similar, para introducir la probabilidad de p\'erdida de paquetes modificamos el mismo m\'etodo ($'send\_and\_queue'$) agreg\'andole una condici\'on que chequee si un valor random (entre 0 y 1) es menor a la chance de perder el paquete ($ACK\_chance$, tambi\'en determinado por el cliente antes de enviar datos al servidor). De ser as\'i, el m\'etodo no llama a la funci\'on $send$ de Socket y se encola para ser retransmitido en un futuro.

Por otro lado, como el objetivo del trabajo pr\'actico es observar qu\'e valores de $\alpha$ y $\mathcal{B}$ optimizan el c\'aluclo de RTO, introducimos la posibilidad de cambiar estos par\'ametros antes de comenzar la transmisi\'on de paquetes entre el cliente y el servidor. De nuevo, el cliente es el encargado de hacerlo. Al momento de llamar al m\'etodo 'run' se le pasan ambos par\'ametros, los cuales son considerados cuando en $rto.py$ se llama al m\'etodo $update\_rtt\_estimation\_with$.

Para realizar las experimentaciones, que detallaremos en la secci\'on siguiente, haremos uso del script $run.sh$, en el cual seteamos los par\'ametros de delay, probabilidad de p\'erdidad de paquetes, $\alpha$ y $\mathcal{B}$. Luego, en el mismo script, abrimos el servidor, conectamos el cliente y enviamos un archivo del cliente al servidor con los par\'ametros previamente mencionados. Volcamos los resultados de cada experimento en el directorio $experimentos/$. En esta etapa, recolectaremos informaci\'on de todo tipo como por ejemplo: variaci\'on de RTT y RTO a lo largo de la transmisi\'on, cantidad de retransmisiones, RTT promedio, RTO promedio, diferencia entre el RTT y RTO, etc. La informaci\'on recolectada es parseada y ploteada con el script $parse-plot.py$